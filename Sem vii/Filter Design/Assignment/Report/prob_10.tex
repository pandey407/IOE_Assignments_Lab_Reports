\problem{What are the advantages of active filter over passive filter? Design an active filter having a
pole at 1000 and a zero at 4000 with a dc gain of 4 using non-inverting configuration.}
The advantages of active filter over a passive filter are compared on the following basis:
\begin{enumerate}
    \item Loading effect: Active filters show no loading effect while passive filters have loading effect. A key advantage of no loading effect on active filters is that higher order filters can be obtained by cascading lower order active filters.
    \item Gain: Unlike passive filters where the gain can be at most 1, active filters can have any value of gain.
    \item Size and weight: Active filters have smaller size and weight since bulky inductors are not used and hence fabrication on smaller chips is possible.
\end{enumerate}
Here,
\begin{fleqn}[\parindent]
   \begin{equation*}
      \begin{split}
         &\text{Pole location } (p)=1000\\
         &\text{Zero location } (z)=4000\\
         &\text{DC gain } (K)=4
         \end{split}
      \end{equation*}
\end{fleqn}
The transfer function can be written as,
\begin{equation*}
    T(s)=K\left(\frac{s+z}{s+p}\right)
\end{equation*}
We have, for realization of $T(s)$ using non inverting configuration shown in Figure~\ref{fig:non-inv},
\begin{equation}
    1+\frac{Z_{f}}{Z_{in}}=K\left(\frac{s+z}{s+p}\right)
    \label{eqn:non-inv}
\end{equation}
\begin{figure}[H]
    \centering
    \fignoninverting
    \caption{Non-inverting configuration}
    \label{fig:non-inv}
\end{figure}
Substituting values in Equation~\ref{eqn:non-inv}, we get,
\begin{equation*}
    \begin{aligned}
        &1+\frac{Z_f}{Z_{in}}=4\left(\frac{s+4000}{s+1000}\right)\\
        &\Rightarrow\frac{Z_f}{Z_{in}}=\frac{4s+16000}{s+1000}-1\\
        &\Rightarrow\frac{Z_f}{Z_{in}}=\frac{4s+16000-s-1000}{s+1000}\\
        &\Rightarrow\frac{Z_f}{Z_{in}}=\frac{3s+15000}{s+1000}\\
        &\Rightarrow\frac{Z_f}{Z_{in}}=\ddfrac{3+\frac{15000}{s}}{1+\frac{1000}{s}}\\
        &\Rightarrow\frac{Z_f}{Z_{in}}=\ddfrac{3+\frac{1}{66.67\times10^{-6}s}}{1+\frac{1}{1\times10^{-3}s}}\\
    \end{aligned}
\end{equation*}
Comparing equivalent terms, we get,
\begin{equation*}
    Z_f=3+\frac{1}{66.67\times10^{-6}s} \quad \quad Z_{in}=1+\frac{1}{1\times10^{-3}s}
\end{equation*}
This can be realized with,
\begin{equation*}
\begin{aligned}
   & R_f=3\text{ }\Omega\quad  &&C_f=66.67 \text{ }\mu \text{F} \\ 
    &R_{in}=1\text{ }\Omega \quad &&C_{in}=1\text{ mF}
\end{aligned}
\end{equation*}
\begin{figure}[H]
    \centering
    \fignoninvertingsolution
    \caption{Designed active filter in non-inverting configuration}
    \label{fig:non-inv-solution}
\end{figure}
These elements can be scaled to practically realizable values as per necessity.