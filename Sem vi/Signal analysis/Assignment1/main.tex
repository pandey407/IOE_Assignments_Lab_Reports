\documentclass{article}[12 pt]

\usepackage{amsmath}
\usepackage{customDecorator}

\begin{document}
\titlePage{Signal Analysis Assignment \#1}{September 1st, 2020}{Dr. Dibakar Raj Panta}
\begin{problem}{
Prove the periodicity condition of continuous exponential signal $x(t)=e^{jw_0t}$.
}
\end{problem}
\begin{solution}
{Here, $x(t)=e^{jw_0t}$, we know that for a continuous time signal $x(t)$ to be periodic with a period $T$, it must comply with $x(t)=x(t+T)$, i.e. the signal should be unaffected by the time shift of period T. \\\\
   i.e. $e^{jw_0t}=e^{jw_0(t+T)}=e^{jw_0t}.e^{jw_0T}$,\\\\so for $x(t)$ to be follow periodicity, the following equation must be true.
   \begin{equation}
       e^{jw_0T}=1
   \end{equation}
   when $w_0=0$, equation (1) holds for all values of $T$,\\
   when $w_0\neq0$, the fundamental period $T_0$ of $x(t)$, i.e. the smallest positive value of T for which equation (1) holds is given by,
   $T_0=\frac{2\pi}{|w_0|}$.\\\\
   This is the condition that a continuous exponential signal $x(t)=e^{jw_0t}$ must hold in order to be periodic.}
\end{solution}

\begin{problem}
{Prove that discrete time complex exponential are periodic only if its frequency is rational.}
\end{problem}

\begin{solution}
{Let us take a discrete time complex exponential signal as $x[n]=e^{jw_0n}$. Considering the discrete time complex exponential with frequency of $w_o + 2\pi$,\\\begin{equation}
    e^{j(w_0+2\pi)n}= e^{jw_0n}.e^{j2\pi n} =e^{jw_0n}
\end{equation}
The analysis of the equation (2) leads us to a conclusion that the signal $x[n]=e^{jw_0n}$ is not distinct like its continuous counterpart, rather the signal with frequency $w_0$ is completely identical to that with frequency $w_0+(2n)\pi$ where $n$ is 1,2,3,\dots. Furthermore, equation (2) also implies the periodicity of a discrete time complex exponential signal in with a period of say, $N$, such that $N>0$.\\\\ For this to be true the signal must follow, $x[n]=x[n+N]$,\\or, $e^{jw_0n}=e^{jw_0(n+N)}=(e^{jw_0n}).(e^{jw_0N})$,\\ so equivalently, for the signal $x[n]=e^{jw_0n}$ to be periodic, $e^{jw_0N}=1$ must be true, which consequently means,\\
    $w_0N=2\pi m$, such that m is an integer.\\\\or,
    \fbox{
    $\frac{w_0}{2\pi}=\frac{m}{N}$
    }. Since m and N both are integers, the ratio $\frac{m}{N}$ is rational.\\\\This means the signal $x[n]=e^{jw_0n}$ is \textbf{periodic only if $\frac{w_0}{2\pi}$, i.e. the frequency of the signal is rational}.}
\end{solution}


\begin{problem}
{Let $x_1(t)=cos6\pi t$ and $x_2(t)=sin30\pi t$. Determine if the function $y=x_1+x_2 $ is periodic, and if it is, find its fundamental period.}
\end{problem}
\begin{solution}
{Here, $x_1(t)=cos6\pi t $, $ x_2(t)=sin30\pi t$ \\\\
     or, $T_1 = \frac{2\pi}{6\pi}=\frac{1}{3}$ and $T_2 = \frac{2\pi}{30\pi} = \frac{1}{15}$,\\\\
     or, $\frac{T_1}{T_2}=\frac{\frac{1}{3}}{\frac{1}{15}}=5$,
    which is a rational number. \\\\Since the ratio $\frac{T_1}{T_2}$ is a rational number such that the two integers (numerator and denominator) are co-prime, \textbf{the sum of the two original periodic signals, $x_1(t)$ and $x_2(t)$ is a period function}.\\\\ Likewise, the fundamental period $T$ can be calculated as,\\
    \fbox{$T=T_1=5T_2=\frac{1}{3}$}
    }
\end{solution}

\begin{problem}
{Determine whether the signal is periodic or aperiodic signal?
\begin{enumerate}
    \item $x(t)=sin(\frac{2\pi}{3}t)$
    \item $x(t)=cos(\frac{\pi}{3}t)+sin(\frac{\pi}{4}t)$
\end{enumerate}
}
\end{problem}
\begin{solution}
{
\begin{enumerate}
    \item Here, $x(t)=sin(\frac{2\pi}{3}t)$,\\\\
    or, $T=\frac{2\pi}{\frac{2\pi}{3}}$=3\\\\A continuous time signal $x(t)$ is said to be periodic if there is a positive a time shift $T$ such that $x(t)=x(t+T)$ for all values of t.\\\\or, $x(t+3)=sin(\frac{2\pi}{3} (t + 3))=sin(\frac{2\pi}{3} t + 2\pi)=sin(\frac{2\pi}{3} t)$\\\\
    Since the given signal complies to the condition for a signal to be periodic,  $x(t)=sin(\frac{2\pi}{3}t)$ \textbf{is periodic}.
    \item Here, $x(t)=cos(\frac{\pi}{3}t)+sin(\frac{\pi}{4}t)$,\\\\Let us represent the signal $x(t)$ as a sum of two signals $x_1(t)$ and $x_2(t)$ such that $x_1(t)=cos(\frac{\pi}{3}t)$ and $x_2(t)=sin(\frac{\pi}{4}t)$.\\\\
    From this, if $T_1$ is the time period of $x_1(t)$ and $T_2$ is that of $x_2(t)$,\\
    $T_1=\frac{2\pi}{\frac{\pi}{3}}=6$ and $T_2=\frac{2\pi}{\frac{\pi}{4}}=8$,\\\\
    so, $\frac{T_1}{T_2}=\frac{6}{8}=\frac{3}{4}$, which is a rational number. \\\\Since the ratio $\frac{T_1}{T_2}$ is a rational number such that the two integers (numerator and denominator) are co-prime,\textbf{ the sum of the two original periodic signals, $x_1(t)$ and $x_2(t)$, is a period function}.\\\\
    The fundamental period of $x(t)$ can be calculated as $T=4T_1=3T_2=24$, so we can conclude that $x(t)=cos(\frac{\pi}{3}t)+sin(\frac{\pi}{4}t)$ is periodic function with a fundamental period of 24.
\end{enumerate}}\end{solution}

\end{document}
