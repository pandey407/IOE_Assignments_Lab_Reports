\documentclass{article}[12 pt]
\usepackage{customDecorator}
\usepackage{amsmath}

\begin{document}
\titlePage{Signal Analysis Assignment \#3}{September 13th, 2020}{Dr. Dibakar Raj Panta}
\begin{problem}
{Find whether the signal is power or energy signal?
\begin{enumerate}
\item 
$x(t)=\begin{cases}
  t & 0 \leq t \leq 1\\
  2-t & 1 \leq t \leq 2\\
  0 & \text{otherwise}
\end{cases}$
\item $x(t)=5cos(\pi t) + sin (5 \pi t)$
\end{enumerate}
}
\end{problem}
\begin{solution}{
The signal energy in the signal $x(t)$ is calculated as,
\begin{equation}
\label{eqn:energysignal}
E=\int_{-\infty}^\infty |x(t)|^2 dt
\end{equation}
Likewise, the signal power in the signal $x(t)$ is calculated as,
\begin{equation}
\label{eqn:powersignal}
P=\lim_{T \to \infty}
\frac{1}{2T}\int_{-T}^T |x(t)|^2 dt
\end{equation}
For a signal $x(t)$ if $0<E<\infty$ then the signal is called an energy signal. For signals that don't satisfy this condition, power is considered and if $0<P<\infty$ then the signal is called a power signal.\\\\If we consider a periodic signal $x(t)$ with period $T_0$, the signal energy in one period is,
\begin{equation}
\label{eqn:energyoneperiod}
E_1=\int_{-\frac{T_0}{2}}^{\frac{T_0}{2}}|x(t)|^2 dt
\end{equation}
So, the energy in n periods is,
\begin{equation}
\label{eqn:energynperiod}
E_n=nE_1=n\int_{-\frac{T_0}{2}}^{\frac{T_0}{2}}|x(t)|^2 dt
\end{equation}
So, the power of this signal over all periods is given as,
\begin{equation}
\label{eqn:powernperiod}
P=\lim_{n \to \infty}\frac{nE_1}{nT_0}=\frac{1}{T_0}\int_{-\frac{T_0}{2}}^{\frac{T_0}{2}}|x(t)|^2 dt
\end{equation}
If the signal energy over one period follows $0<E_1<\infty$, then the total
energy is infinite and the signal power is finite. Therefore, the signal is a power
signal. And if the signal energy in one period is infinite, then both the power and the total
energy are infinite. Consequently, the signal is neither an energy signal nor a
power signal.\\From this, we can draw a conclusion that periodic signals may be power signals but can never be energy signals.
\begin{enumerate}
\item Here, the given signal is mathematically defined as,
\begin{equation}
\label{eqn:questiona}
x(t)=\begin{cases}
  t & 0 \leq t \leq 1\\
  2-t & 1 \leq t \leq 2\\
  0 & \text{otherwise}
\end{cases}
\end{equation}
To find the energy of this signal, we have Eq(\ref{eqn:energysignal}),
\begin{equation*}
\begin{aligned}
E&=\int_{-\infty}^\infty |x(t)|^2 dt\\
&=\int_{-\infty}^0 |x(t)|^2 dt +\int_{0}^1 |x(t)|^2 dt + \int_{1}^2 |x(t)|^2 dt +\int_{2}^\infty |x(t)|^2 dt\\
&=\int_{0}^1 |x(t)|^2 dt + \int_{1}^2 |x(t)|^2 dt && \textbf{(From Eq.(\ref{eqn:questiona}))}\\
&=\int_{0}^1 t^2 dt + \int_{1}^2 (2-t)^2 dt\\
&=\left[\frac{t^3}{3}\right]_0^1+\int_{1}^2 (4-4t+t^2) dt\\
&=\frac{1}{3}+4\int_{1}^2 dt-4\int_{1}^2 t dt+\int_{1}^2 t^2 dt\\
&=\frac{1}{3}+4\left[t\right]_1^2-4\left[\frac{t^2}{2}\right]_1^2+\left[\frac{t^3}{3}\right]_1^2\\
&=\frac{1}{3}+4-4\left[\frac{4}{2}-\frac{1}{2}\right]+\frac{8}{3}-\frac{1}{3}\\
&=\frac{2}{3}\text{ Joules}
\end{aligned}
\end{equation*}
The value of $E$ satisfies the condition, $0<E<\infty$, i.e. it is non-zero and finite. The time period of $x(t)$ is $\infty$ since it is an aperiodic signal. From Eq.({\ref{eqn:powersignal}}), the power of the signal is 0.\\
Thus, \textbf{signal $x(t)$ is an energy signal}.
\item Here, the signal $x(t)=5cos(\pi t) + sin (5 \pi t)$ can be represented as the sum of two signals, $x_1(t)=5cos(\pi t)$ and $x_2(t)=sin (5 \pi t)$. If $T_1$ and $T_2$ be the periods of $x_1(t)$ and $x_2(t)$ respectively, then,\\
$T_1=\frac{2\pi}{\pi}=2$ and $T_2=\frac{2\pi}{5 \pi}=\frac{2}{5}$,\\\\
    so, $\frac{T_1}{T_2}=\frac{2}{\frac{2}{5}}=5$, which is a rational number. \\\\Since the ratio $\frac{T_1}{T_2}$ is a rational number such that the two integers (numerator and denominator) are co-prime, the sum of the two original periodic signals, $x_1(t)$ and $x_2(t)$, is a period function.The fundamental period of $x(t)$ can be calculated as $T_0=T_1=5T_2=2$, so we can say that $x(t)=5cos(\pi t) + sin (5 \pi t)$ \textbf{is periodic function with a fundamental period of 2}.\\\\Since the signal is periodic, it can't be an energy signal, so we determine the power of the signal to check if it's a power signal.\\From Eq.(\ref{eqn:powernperiod}), for $T_0=2$, we have, 
    \begin{equation}
    \label{eqn:solutionb}
P=\frac{1}{2}\int_{-1}^{1}|x(t)|^2 dt=\frac{1}{2}P'\\
\end{equation}
Solving only the integral part from Eq.(\ref{eqn:solutionb}),
\begin{equation*}
\begin{aligned}
P'&=\int_{-1}^1 |x(t)|^2 dt\\
&=\int_{-1}^1 (5cos(\pi t) + sin (5 \pi t))^2dt\\
&=\int_{-1}^1 (25cos^2(\pi t) + 10cos(\pi t) sin (5 \pi t)+sin^2 (5 \pi t))dt
\end{aligned}
\end{equation*}
On solving the integral, we get, $P'=[f(t)]_{-1}^1$, where,
\begin{equation}
\label{eqn:ft}
f(t)=\left[-\frac{3\sin\left(10{\pi}t\right)+50\cos\left(6{\pi}t\right)+75\cos\left(4{\pi}t\right)-375\sin\left(2{\pi}t\right)-780{\pi}t}{60{\pi}}\right]
\end{equation}
From Eq.(\ref{eqn:ft}), we can find, 
\begin{equation*}
\begin{aligned}
f(1)&=\frac{-125+780 \pi }{60 \pi}\\
f(-1)&=\frac{-780 \pi-125}{60 \pi}\\\\
\Rightarrow P'&=f(1)-f(-1)\\
&=\frac{-125+780 \pi }{60 \pi}-\frac{-780 \pi-125}{60 \pi}\\
&=\frac{1560 \pi}{60 \pi}\\
P'&=26
\end{aligned}
\end{equation*}
Putting the value $P'=26$ in Eq.(\ref{eqn:solutionb}), we get, 
\begin{equation*}
P=\frac{26}{2}=13\text{ Watt}
\end{equation*}
The power $P$ satisfies the condition, $0<P<\infty$, i.e. non-zero and finite. Thus, \textbf{signal $x(t)$ is a power signal}.
\end{enumerate}
}
\end{solution}
\begin{problem}{Explain example 1.6 given in \textit{Signals and Systems by Alan V. Oppenheim, Alan S.Willsky, with S.Hamid} in your own words.\\Determine the fundamental period of the discrete-time signal\begin{equation}
\label{eqn:question2}
x[n]=e^{j(\frac{2 \pi}{3})n}+e^{j(\frac{3 \pi}{4})n}
\end{equation}
}
\end{problem}
\begin{solution}{An exponential signal in discrete-time, say, $x[n]=e^{jw_o n}$ is periodic with fundamental period $N$ only if $x[n]=x[n+N]$. Further evaluation of this gives us a relation,
\begin{equation*}
N=m\left(\frac{2\pi}{w_o}\right)\text{, where m is an integer}
\end{equation*}
We could figure out the fundamental period of the signal $x[n]$ using this relation, but a simpler approach would be to determine the perodic occurance of the values based on the value of $w_0$. If we write the signal $x[n]$ from Eq.(\ref{eqn:question2}) as $x[n]=x_1[n]+x_2[n]$ where, $x_1[n]=e^{j(\frac{2 \pi}{3})n}$ and $x_2[n]=e^{j(\frac{3 \pi}{4})n}$, then we can easily analyse the identical values of $x_1[n]$ and $x_2[n]$ that repeat at integer multiple of $2\pi$ angle to determine the fundamental period of $x[n]$.\\\\For $x_1[n]$, if $n$ is incremented by a value of 3, the signal repeats at $2\pi$ interval and the angle will be incremented by a single multiple of $2\pi$ for every single multiple increment of 3 in $n$. Thus we can say that the fundamental period of the signal $x_1[n]$ is 3.\\\\Similarly, for the signal $x_2[n]$ if the value of $n$ is incremented by $\frac{8}{3}$, the value of angle is $2\pi$,but a fractional increment isn't possible in $n$ since it is an integer. Again, if the value of angle is the 2nd multiple of $2 \pi$, i.e. $4 \pi$, the value of $n$ must be incremented by $\frac{16}{3}$, which is also an invalid entry. The next multiple of $2\pi$, i.e. $6\pi$ can be attained by an increment of 8 in n, which is a valid increment. So every 8 increment in $n$ produces an integer multiple of $6\pi$, thus the fundamental period of $x_2[n]$ is 8.\\\\Now, for the signal $x[n]$ to be periodic, both $x_1[n]$ and $x_2[n]$ must go through an integer multiple of it's own fundamental period. This can be simply calculated as the L.C.M. of the two fundamental periods, 3 and 8, i.e.  24. So the signal $x_1[n]$ would've gone through it's 8th fundamental period and $x_2[n]$ it's 3rd fundamental period when the signal $x[n]$ completes one of its fundamental period of 24.}
\end{solution}
\end{document}