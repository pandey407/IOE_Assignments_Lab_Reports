\documentclass{home_assignment}
\usepackage{multicol}
\usepackage{tikz}

\usepackage{multirow}
\usepackage{diagbox}
\usepackage{pgfplots}
\pgfplotsset{compat=1.17}

\newcolumntype{C}[1]{>{\centering\arraybackslash}m{#1}}


\newcommand{\magplot}{
\begin{tikzpicture}
  \begin{axis}[
      axis x line=middle,
      axis y line=none,
      xmin=-5,xmax=5,
      ymin=0,ymax=2,
      xlabel=$k$,
      xlabel style = {anchor=west},
      xtick={-4,-3,...,0,...,3,4}
    ]
    info{
    \node at (0,1.3) [anchor=south] {$|a_k|$};
    };
    \addplot[thick,black,mark=none,] coordinates {(0, 0) (0, 1)};
    \addplot[thick,black,mark=none,] coordinates {(1, 0) (1, 1.118033989)};
    \addplot[thick,black,mark=none,] coordinates {(-1, 0) (-1, 1.118033989)};
    \addplot[thick,black,mark=none,] coordinates {(2, 0) (2,0.5)};
    \addplot[thick,black,mark=none,] coordinates {(-2, 0) (-2,0.5)};
  \end{axis}
\end{tikzpicture}
}
\newcommand{\phaseplot}{
\begin{tikzpicture}
  \begin{axis}[
      axis x line=middle,
      axis y line=none,
      xmin=-5,xmax=5,
      ymin=-2,ymax=2,
      xlabel=$k$,
      xlabel style = {anchor=west},
      xtick={-4,-3,-1,0,2,3,4}
    ]
    info{
    \node at (0,1) [anchor=south] {$\measuredangle a_k$};
    \node at (-2,0) [anchor=south] {-2};
    \node at (1,0) [anchor=south] {1};
    };
    \addplot[thick,black,mark=none,] coordinates {(1, 0) (1, -0.463647609)};
    \addplot[thick,black,mark=none,] coordinates {(-1, 0) (-1, 0.463647609)};
    \addplot[thick,black,mark=none,] coordinates {(2, 0) (2,0.7853981634)};
    \addplot[thick,black,mark=none,] coordinates {(-2, 0) (-2,-0.785398163)};
  \end{axis}
\end{tikzpicture}
}

\begin{document}
\titlePage{Signal Analysis Assignment \#4}{September 28, 2020}{Dr. Dibakar Raj Panta}
\problem{Prove that the following identity holds:
\begin{equation*}
\int_T{e^{j(k-n)\omega_0t}dt}=\begin{cases}
T, & \text{for }k=n\\
0, & \text{otherwise}
\end{cases}
\end{equation*}
}
\solution{Here, we need to solve the integral on the LHS, i.e. \begin{equation}
\int_T{e^{j(k-n)\omega_0t}dt}
\label{eq:prob1}
\end{equation}
In Eq.(\ref{eq:prob1}) we are concerned with integrating $e^{j(k-n)\omega_0t}$ over an interval of length T. So, we will obtain the same result over any interval of length T, say, $[0,T], [\frac{-T}{2},\frac{T}{2}], [T,2T]$ and so on. Choosing $[0,T]$ for simplicity, results in the integral, 
\begin{equation}
\int_0^T{e^{j(k-n)\omega_0t}dt}=\int_0^T{cos[(k-n)\omega_0t]dt+jsin[(k-n)\omega_0t]dt}
\label{eq:prob1solution}
\end{equation}
For $k=n$, the Eq.(\ref{eq:prob1solution}) becomes, 
\begin{equation*}
\begin{aligned}
&=\int_0^T{cos(0)dt}+j\int_0^T{sin(0)dt}\\
&=\int_0^T{dt}+j.0=T
\end{aligned}
\end{equation*}
For $k\neq n$, $cos[(k-n)\omega_0t]$ and $sin[(k-n)\omega_0t]$ are periodic sinusoidal functions with a fundamental period of $\left|\frac{T}{k-n}\right|$. Since we are integrating the Eq.(\ref{eq:prob1solution}) over an interval of length T, such that the interval is an integral number of periods of the signal, i.e. the interval is $(k-n)^{th}$ multiple of the periods of the signal. This way, the integration can be represented as a measure of the total area under the functions over the interval T, hence the integral for $k\neq n$ results in 0.\\
The overall evaluation of the integral represented by Eq.(\ref{eq:prob1}) shows that,
\begin{equation*}
\int_T{e^{j(k-n)\omega_0t}dt}=\begin{cases}
T, & \text{for }k=n\\
0, & \text{otherwise}
\end{cases}
\end{equation*}
}
\problem{Determine the complex form of Fourier series from its trigonometric form and vice-versa.}
\solution{
\subsection*{Complex Form from Trigonometric Form}
The trigonometric form of fourier series can be written as,
\begin{equation}
x(t)=a_0+\sum_{n=1}^{\infty}a_ncos(n\omega_0 t)+\sum_{n=1}^{\infty}b_nsin(n\omega_0 t)
\label{eq:prob2trigono}
\end{equation}
where,
\begin{equation*}
\begin{aligned}
a_0&=\frac{1}{T}\int_{-T/2}^{T/2}{x(t)dt}\\
a_n&=\frac{2}{T}\int_{-T/2}^{T/2}{x(t)cos(n\omega_0 t)dt}\\
b_n&=\frac{2}{T}\int_{-T/2}^{T/2}{x(t)sin(n\omega_0 t)dt}
\end{aligned}
\end{equation*}
From Euler's formulae, we have,
\begin{equation*}
cos(n\omega_0 t)=\frac{e^{jn\omega_0 t}+e^{-jn\omega_0 t}}{2} \text{, } sin(n\omega_0 t)=\frac{e^{jn\omega_0 t}-e^{-jn\omega_0 t}}{2j} =-j\left(\frac{e^{jn\omega_0 t}-e^{-jn\omega_0 t}}{2}\right)
\end{equation*}
Using this, we can reduce the Eq.(\ref{eq:prob2trigono}) as,
\begin{equation*}
\begin{aligned}
x(t)&= a_0+\sum_{n=1}^{\infty}a_n\left(\frac{e^{jn\omega_0 t}+e^{-jn\omega_0 t}}{2}\right)+\sum_{n=1}^{\infty}b_n\left(\frac{-je^{jn\omega_0 t}+je^{-jn\omega_0 t}}{2}\right)\\
&=a_0+\sum_{n=1}^{\infty}\left(\frac{a_n-jb_n}{2}\right)e^{jn\omega_0 t}+\sum_{n=1}^{\infty}\left(\frac{a_n+jb_n}{2}\right)e^{-jn\omega_0 t}\\
&=a_0+\sum_{n=1}^{\infty}\left(\frac{a_n-jb_n}{2}\right)e^{jn\omega_0 t}+\sum_{n=-\infty}^{-1}\left(\frac{a_{-n}+jb_{-n}}{2}\right)e^{jn\omega_0 t}\\
&=C_0+\sum_{n=1}^{\infty}C_ne^{jn\omega_0 t}+\sum_{n=-\infty}^{-1}C_{n}^*e^{jn\omega_0 t}
\end{aligned}
\end{equation*}
where, $C_0=a_0$, $C_n=\left(\frac{a_n-jb_n}{2}\right)$ and $C_{n}^*=\left(\frac{a_{-n}+jb_{-n}}{2}\right)$ such that $C_n$ and $C_{n}^*$ are complex conjugates. If we choose to represent all the complex fourier coefficients with $C_n$ such that $C_n$ is the complex conjugate of $C_{-n}$ then, \textbf{the complex form of fourier series is given by,}
$$x(t)=\sum_{-\infty}^{\infty}C_ne^{jn\omega_0 t}$$
\subsection*{Trigonometric Form from Complex Form}
The complex form of fourier series can be written as,
\begin{equation}
x(t)=\sum_{-\infty}^{\infty}C_ne^{jn\omega_0 t}
\label{eq:pro2complex}
\end{equation}
Eq.(\ref{eq:pro2complex}) can be expanded as the sum,
\begin{equation*}
\begin{split}
x(t)&=C_0+C_1e^{j\omega_0 t}+C_2e^{2j\omega_0 t}+\dots+C_ne^{jn\omega t}+\dots+C_{-1}e^{-j\omega_0 t}+C_{-2}e^{-2j\omega_0 t}+\dots+C_{-n}e^{-jn\omega_0 t}+\dots\\
&=C_0+C_1[cos(\omega_0 t)+jsin(\omega_0 t)]+C_2[cos(2\omega_0 t)+jsin(2\omega_0 t)]+\dots+C_n[cos(n\omega_0 t)+jsin(n\omega_0 t)]+\dots+\\&C_{-1}[cos(\omega_0 t)-jsin(\omega_0 t)]+C_{-2}[cos(2\omega_0 t)-jsin(2\omega_0 t)]+\dots+C_{-n}[cos(n\omega_0 t)-jsin(n\omega_0 t)]+\dots
\\&=C_0+(C_1+C_{-1})cos(\omega_0 t)+(C_2+C_{-2})cos(2\omega_0 t)+\dots+(C_n+C_{-n})cos(n\omega_0 t)+\dots+\\&(C_1-C_{-1})sin(\omega_0 t)+j(C_2-C_{-2})sin(2\omega_0 t)+\dots+j(C_n-C_{-n})sin(n\omega_0 t)+\dots\\
&=C_0+\sum_{n=1}^{\infty}(C_n+C_{-n})cos(n\omega_0 t)+\sum_{n=1}^{\infty}j(C_n-C_{-n})sin(n\omega_0 t)
\end{split}
\end{equation*}
This can be rearranged such that \textbf{the fourier series in trigonometric form is given by,} 
$$
x(t)=a_0+\sum_{n=1}^{\infty}a_ncos(n\omega_0 t)+\sum_{n=1}^{\infty}b_nsin(n\omega_0 t)
$$
 where,
\begin{equation*}
\begin{aligned}
a_0&=C_0\\
a_n&=(C_n+C_{-n})\\
b_n&=j(C_n-C_{-n})
\end{aligned}
\end{equation*}
}
\problem{Plot the graph of the magnitude and phase of $a_k$ for the example below and interpret.}
\solution{Here, $x(t)=1+sin(\omega_0 t)+2cos(\omega_0 t)+cos\left(2\omega_0 t+\frac{\pi}{4}\right)$, 
which has the fundamental frequency $\omega_0$. The signal $x(t)$ can be expanded directly in terms of its complex exponentials for easier approach to determine the complex fourier coefficients.\\
Using the euler's formulae, we get,
\begin{equation*}
\begin{aligned}
x(t)&=1+\frac{1}{2j}[e^{j\omega_0 t}-e^{-j\omega_0 t}]+[e^{j\omega_0 t}+e^{-j\omega_0 t}]+\frac{1}{2}[e^{j(2\omega_0 t+\frac{\pi}{4})}-e^{-j(2\omega_0 t+\frac{\pi}{4})}]\\&=1+\left(1+\frac{1}{2j}\right)e^{j\omega_0 t}+\left(1-\frac{1}{2j}\right)e^{-j\omega_0 t}+\left(\frac{1}{2}e^{j(\frac{\pi}{4})}\right)e^{2j\omega_0 t}+\left(\frac{1}{2}e^{-j(\frac{\pi}{4})}\right)e^{-2j\omega_0 t}
\end{aligned}
\end{equation*}
From this, we can determine the fourier series coefficients as,
\begin{multicols}{2}
\noindent
\begin{equation*}
\begin{aligned}
a_0&=1,\\
a_1&=\left(1+\frac{1}{2j}\right)=\left(1-\frac{1}{2}j\right),\\
a_{-1}&=\left(1-\frac{1}{2j}\right)=\left(1+\frac{1}{2}j\right),
\end{aligned}
\end{equation*}
\begin{equation*}
\begin{aligned}
a_2&=\frac{1}{2}e^{j(\frac{\pi}{4})}=\frac{\sqrt{2}}{4}(1+j),\\
a_{-2}&=\frac{1}{2}e^{-j(\frac{\pi}{4})}=\frac{\sqrt{2}}{4}(1-j),\\
a_k&=0,|k|>2
\end{aligned}
\end{equation*}
\end{multicols}
To plot the magnitude and phase of $a_k$, the following table will be used,
\begin{table}[H]
\centering
\begin{tabular}{|l|C{3cm}|C{3cm}|}
\hline
\diagbox{$a_k$ for }{Value} & Magnitude ($|a_k|$)& Phase($\measuredangle a_k$)\\
\hline
$k=0$ & 1 & 0\\
\hline
$k=1$ & \multirow{ 2}{*}{1.118033989}& -0.463647609\\
\cline{0-0}\cline{3-3}
$k=-1$ & &0.463647609\\
\hline
$k=2$ &  \multirow{ 2}{*}{0.5} & 0.7853981634\\
\cline{0-0}\cline{3-3}
$k=-2$ &  & -0.7853981634\\
\hline
$|k|>2$ & 0 & 0\\
\hline
\end{tabular}
 \caption{Magnitude and Phase calculations for $a_k$}
\label{tab:schedule_list}
\end{table}
\begin{figure}[H]
\centering
\magplot
\caption{Plot for magnitude of $a_k$}
\label{fig:magnitude}
\end{figure}
\begin{figure}[H]
\centering
\phaseplot
\caption{Plot for phase of $a_k$}
\label{fig:phase}
\end{figure}
From Figure(\ref{fig:magnitude}) and (\ref{fig:phase}), viz. the magnitude and phase plots of $a_k$, i.e. the complex coefficients of fourier series, we can see that the plots are symmetrically distributed. We can draw out the following properties of the fourier series spectrum $a_k$,
\subsubsection*{Property 1} 
If $x(t)$ is a real-valued periodic signal, then, $a_k={a_{-k}}^*$, i.e. the fourier coefficients follow \textbf{conjugate symmetry}. This is evident from the values of $a_k$ determined above.
\subsubsection*{Property 2}
If $x(t)=x(-t)$, i.e. the signal has even-symmetry about the origin, then, $a_k=a_{-k}$. This is true since the fourier coefficients of even signals are real-valued and the fourier expansion of a real-valued fourier coefficient results in only the cosine terms, which is the simplest form of an even signal. 
\subsubsection*{Property 3}
If $x(t)=-x(-t)$, i.e. the signal has odd-symmetry about the origin, then, $a_k=-a_{-k}$. This is true since the fourier coefficients of odd signals are purely imaginary and the fourier expansion of a purely imaginary fourier coefficient results in only the sine terms, which is the simplest form of an odd signal. 
}
\end{document}
